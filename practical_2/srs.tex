\documentclass{scrreprt}
\usepackage{listings}
\usepackage{underscore}
\usepackage{bbding}  % to use custom itemize font
\usepackage{titling}  % to use \theauthor (author and title name seperately)
\usepackage[shortlabels]{enumitem}
\usepackage[utf8]{inputenc}
\usepackage[english]{babel}
\usepackage{graphicx, subfig, graphics, flowchart}
\usepackage{float}  % perfect fit graphic with command [H]
\usepackage{array}  % for list of figures
\graphicspath{ {figures/} }  % define graphicspath list out tabel of figures
\usepackage{hyperref}
% set up
\hypersetup{
    bookmarks=false,    % show bookmarks bar?
    pdftitle={Software Requirement Specification},    % title
    pdfauthor={Gahan Saraiya},                     % author
    pdfsubject={SRS for SaaS (generic web portal)}, % subject of the document
    pdfkeywords={TeX, LaTeX, graphics, images, generic, web portal, SaaS, PaaS}, % list of keywords
    colorlinks=true,       % false: boxed links; true: colored links
    linkcolor=blue,       % color of internal links
    citecolor=black,       % color of links to bibliography
    filecolor=black,        % color of file links
    urlcolor=purple,        % color of external links
    linktoc=page            % only page is linked
}

% alterante itemize command
\newcommand*\tick{\item[\Checkmark]}
\newcommand*\arrow{\item[$\Rightarrow$]}
\newcommand*\fail{\item[\XSolidBrush]}
\def\myversion{1.0 }
\def\university{Institute of Technology, \\Nirma University}
\def\superuser{Platform Administrator}
\def\admin{Administrator}
\def\staff{Staff user}
\def\user{End User}
\date{}
\title{Generic Web Portal (SaaS)}
\author{Gahan Saraiya}
%Load nomenclature and glossary files
%\loadglsentries{glossary}
\begin{document}

\begin{flushright}
    \rule{16cm}{5pt}\vskip1cm
    \begin{bfseries}
        \Huge{SOFTWARE REQUIREMENTS\\ SPECIFICATION}\\
%        \vspace{1.9cm}
        for\\
        \vspace{1.9cm}
        \thetitle\\
        \vspace{1.9cm}
        \LARGE{Version \myversion}\\
        \vspace{1.9cm}
        Prepared by \theauthor (18MCEC10)\\
        \vspace{1.9cm}
        \university\\
        \vspace{1.9cm}
        \today\\
    \end{bfseries}
\end{flushright}

\tableofcontents  % list out index for all chapters
\listoffigures  % list out index for figures

\chapter*{Revision History}
\begin{center}
    \begin{tabular}{|c|c|c|c|}
        \hline
	    Date & Revision & Reason For Changes & Author \\
        \hline
	     & 1.0 & Initial Version & \theauthor\\
        \hline
    \end{tabular}
\end{center}

\chapter{Introduction}
\label{introduction}

\section{Purpose}
\label{purpose}
This project targets the generic implementation of web portal so that any user can deploy their client-server architecture.
The project is capable of deploying any e-commerce/management system with just following few simple steps in wizard.

\section{Project Scope}
\label{scope}
This Generic Web Portal is a "Software as a service" for user or a company who wants to launch their e-commerce site or a management portal for item/product sales, purchase, accounting etc. This generic service helps stakeholders to launch it in just few steps such as some required information, policy agreement and payment to launch the desire service platform of their own. \\
Here to make the system generic all the products are treated like object as if in book store books are considered as  objects having their properties unique (ISBN) and non-unique(Author, Title, Publisher, etc). similarly if it is for hotel management system rooms are treated as an object having their unique properties like room number or assigned id and non-unique properties such as availability, facility, floor, etc. Hence just by interpreting each product of real world as an object and treating it's use as a transaction.
\\
Also the system is meant to follow disciplined agile delivery such that it will be compatible to adapt new requirements and/or changes with the help of support option such that a company or user can deploy a new system of their own instead of using generic platform.

\section{Definitions, acronyms, and abbreviations}
\label{acronyms}
\begin{center}
\begin{tabular}{ |p{3cm}||p{9cm}|  }
	\hline
	\multicolumn{2}{|c|}{Definitions, acronyms, and abbreviations} \\
	\hline
	Acronym		&	Meaning	\\
	\hline
	SRS			&   Software Requirements Specification	\\
	SaaS		&   Software as a Service	\\
	PaaS		&	Platform as a Service	\\
	Stakeholder	&	Refers to a client or company	\\
	OWASP		&	Open Web Application Security Project	\\
	API			&	Application Programming Interface	\\

	\hline
\end{tabular}
\end{center}

\chapter{Overall Description}

\section{Product Perspective}
\label{product_perspective}
The diagram below shows the perspective of a system.
\begin{figure}[H]
\centering	
\includegraphics[scale=0.3]{perspective.png}
\caption{Product Perspective}
\end{figure}
When a user visits the platform on web, the task for user is to just decide their requirement and choose suitable management portal and it will be deployed by just following simple steps includes information fill up and payment service and user verification

\section{Product Features}
\begin{enumerate}
	\item Easy to deploy web service on SaaS
	\item Isolation for different user
	\item Generic Web Platform to deploy various kind of web service i.e. accounting, inventory management, hotel management, hospital management or any kind of product or utility management system
	\item To be known and marketed as PaaS and SaaS
	\item System is scalable up to simple accounting management to e-commerce platform
	\item Dashboard for analytics and database management
	\item Various set of permission management
	\item follows guideline of OWASP guidelines
	\item Easy to consume APIs
\end{enumerate}

\section{User Classes and Characteristics}
There are three kind of users utilizing the system:
\paragraph[developerRole]{Developer} Developer can be individual or part of a team or company accessing the system to deploy their web service
\paragraph[endUser]{End User} end users access the system to interact with already deployed web service by Developer

\section{Operating Environment}
The Platform is deployed on centralized server hence any user having a web browser and active Internet connection can interact with this portal.

\section{Design and Implementation Constraints}
\begin{itemize}
	\item However user can interact with any web browser which is constrained by hardware capability and operating system. Hence It is require to have enough free memory and storage space for web browser on the device. 
	\item Also it is recommended that user should use latest available web browser.
	\item Depending on web browser type and version user interface and interaction ease may slightly vary.
	\item Performance of page load are also depended upon user's bandwidth too and user needs to have a stable Internet connection with good speed
\end{itemize}


\section{User Documentation}
\begin{itemize}
	\item API documentation for developer
	\item online help and support for deployment
	\item video tutorial to show a demo of deployment
\end{itemize}
Rest of related document for end-user needs to be prepared by the web service owner/developer team whoever deployed the web service.

\section{Assumptions and Dependencies}
The Platform has only term agreement with the owner of web service. As the end-user (if any) will be consuming services provided by respective web service owner and for any help and support related to web service user needs to follow contact to support team of respective web service. i.e. if a end-user is accessing hotel management service deployed by a company XYZ and has any query or issue related to the system then this end-user needs to contact support staff of company XYZ only.

User may need to allow executing javascripts and image to experience the system else the page may not work as expected or may not work at all.
The end-user might have to allow additional access to the camera or flash plugin or any other plugin or add on if it is require by web service owner.
 

\chapter{External Interface Requirements}

\section{User Interfaces}
\begin{itemize}
	\item Admin Panel
	\item Dashboard
	\item User view
	\item Critical Section (to modify database of their system)
\end{itemize}

\section{Software Interfaces}
\begin{itemize}
	\item Database: SQL Server
	\item HTTP connection: Apache 2 (to serve request)
	\item root access: SSH shell
	\item version control: git
	\item Technology: Python3, Django Framework, HTML, CSS, js
\end{itemize}

\section{Communications Interfaces}
\begin{itemize}
	\item Require Internet Connection
	\item client-server communication established over HTTP
\end{itemize}

\chapter{System Features}
\section{Functional Requirements}
\label{functions}
\subsection{\admin's Requirement}
\begin{enumerate}[start=1,label={\bfseries REQ \arabic*:}]
	\addtolength{\itemindent}{40pt}
	\item Login
		\\To login into system valid email address and strong password is required
	\item Sign Up
		\\ \admin\space can register to deploy the web service which require below basic information:
		\begin{itemize}
			\arrow Valid Email address
			\arrow Strong Password
			\arrow Name
			\arrow Mobile Number
			\arrow Select service plan
			\arrow Mailing address
			\arrow payment
		\end{itemize}
		Upon successful registration record will be stored in database and email verification link will be sent to register email address.
	\item Profile Management
		\\ \admin s can update their profile information anytime
		\\All fields except email address are updatable
		\\Changes will be reflected in database when \admin \space updates the information
		\\ \admin can also add other optional field information such as alternate address, contact number, profile image
	\item Customize Database/service dashboard
		\\ \admin \space can follow documentation to add/change existing database fields or tables
	\item Credential Recovery
		\\If \admin \space is not remembering password for accessing system a password reset email will be send if require 
	\item Create staff user
		\\At organization level there might be one or more people assigned the different task to manage and for that purpose sharing the ownership credential is risky as well as hard to track activity of each staff hence \admin \space can create \staff \space and assign them different accessible permission to the dashboard
	\item Activity Log
		\\All the sensitive dashboard activity logged 
	\item Report Generation
		\\User can generate report in json or pdf format for record(s)
	\item Invoice Generation
		\\For any sale record Taxable Invoice can be generated in pdf format
	\item Support and Maintenance
		\\For any trouble or inquiry \admin \space can contact support team
	
\end{enumerate}
\subsection{\user's Requirement}
\begin{enumerate}[start=1,label={\bfseries REQ \arabic*:}]
	\addtolength{\itemindent}{40pt}
	\item Login
		\\To login into system valid email address and strong password is required
	\item Sign Up
		\\ \user \space can register to deploy the web service which require below basic information:
			\begin{itemize}
				\arrow Valid Email address
				\arrow Strong Password
				\arrow Name
				\arrow Mobile Number
				\arrow Select service plan
				\arrow Mailing address
				\arrow payment
			\end{itemize}
		Upon successful registration record will be stored in database and email verification link will be sent to register email address.
	\item Profile Management
		\\ \user s can update their profile information anytime
		\\All fields except email address are updatable
		\\Changes will be reflected in database when \user \space updates the information
		\\ \user can also add other optional field information such as alternate address, contact number, profile image
	\item Customize Database/service dashboard
		\\ \user \space can follow documentation to add/change existing database fields or tables
	\item Credential Recovery
		\\If \user \space is not remembering password for accessing system a password reset email will be send if require 
\end{enumerate}

\chapter{Other Nonfunctional Requirements}

\section{Performance Requirements}
$<$If there are performance requirements for the product under various 
circumstances, state them here and explain their rationale, to help the 
developers understand the intent and make suitable design choices. Specify the 
timing relationships for real time systems. Make such requirements as specific 
as possible. You may need to state performance requirements for individual 
functional requirements or features.$>$

\section{Safety Requirements}
$<$Specify those requirements that are concerned with possible loss, damage, or 
harm that could result from the use of the product. Define any safeguards or 
actions that must be taken, as well as actions that must be prevented. Refer to 
any external policies or regulations that state safety issues that affect the 
product’s design or use. Define any safety certifications that must be 
satisfied.$>$

\section{Security Requirements}
$<$Specify any requirements regarding security or privacy issues surrounding use 
of the product or protection of the data used or created by the product. Define 
any user identity authentication requirements. Refer to any external policies or 
regulations containing security issues that affect the product. Define any 
security or privacy certifications that must be satisfied.$>$

\section{Software Quality Attributes}
$<$Specify any additional quality characteristics for the product that will be 
important to either the customers or the developers. Some to consider are: 
adaptability, availability, correctness, flexibility, interoperability, 
maintainability, portability, reliability, reusability, robustness, testability, 
and usability. Write these to be specific, quantitative, and verifiable when 
possible. At the least, clarify the relative preferences for various attributes, 
such as ease of use over ease of learning.$>$

\section{Business Rules}
$<$List any operating principles about the product, such as which individuals or 
roles can perform which functions under specific circumstances. These are not 
functional requirements in themselves, but they may imply certain functional 
requirements to enforce the rules.$>$


\chapter{Other Requirements}
$<$Define any other requirements not covered elsewhere in the SRS. This might 
include database requirements, internationalization requirements, legal 
requirements, reuse objectives for the project, and so on. Add any new sections 
that are pertinent to the project.$>$

\section{Appendix A: Glossary}
%see https://en.wikibooks.org/wiki/LaTeX/Glossary
$<$Define all the terms necessary to properly interpret the SRS, including 
acronyms and abbreviations. You may wish to build a separate glossary that spans 
multiple projects or the entire organization, and just include terms specific to 
a single project in each SRS.$>$

\section{Appendix B: Analysis Models}
$<$Optionally, include any pertinent analysis models, such as data flow 
diagrams, class diagrams, state-transition diagrams, or entity-relationship 
diagrams.$>$

\section{Appendix C: To Be Determined List}
$<$Collect a numbered list of the TBD (to be determined) references that remain 
in the SRS so they can be tracked to closure.$>$

\end{document}
